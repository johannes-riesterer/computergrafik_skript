\documentclass[a4paper,13pt]{scrartcl}

\usepackage[latin1]{inputenc}
\usepackage{amsfonts}
\usepackage{amsmath}
\usepackage{amssymb}
\usepackage{amsthm}
\usepackage{color}
\usepackage[ngerman]{babel}
\usepackage[pdftex]{graphicx}
%\usepackage[T1]{fontenc}
\usepackage{graphicx}
\pagestyle{empty}

%\topmargin20mm
\oddsidemargin0mm
\parindent0mm
\parskip2mm
\textheight25.5cm
\textwidth15.8cm
\unitlength1mm


% Listings
\usepackage{listings}
\definecolor{Brown}{cmyk}{0,0.81,1,0.60}
\definecolor{OliveGreen}{cmyk}{0.64,0,0.95,0.40}
\definecolor{CadetBlue}{cmyk}{0.62,0.57,0.23,0}
\definecolor{lightlightgray}{gray}{0.9}
\lstset{
    language=C,                             % Code langugage
    basicstyle=\rm\ttfamily,                % Code font, Examples: \footnotesize, \ttfamily
    keywordstyle=\color{OliveGreen},        % Keywords font ('*' = uppercase)
    commentstyle=\color{gray},              % Comments font
    numbers=left,                           % Line nums position
    numberstyle=\tiny,                      % Line-numbers fonts
    stepnumber=1,                           % Step between two line-numbers
    numbersep=5pt,                          % How far are line-numbers from code
    backgroundcolor=\color{lightlightgray}, % Choose background color
    frame=none,                             % A frame around the code
    tabsize=2,                              % Default tab size
    captionpos=b,                           % Caption-position = bottom
    breaklines=true,                        % Automatic line breaking?
    breakatwhitespace=false,                % Automatic breaks only at whitespace?
    showspaces=false,                       % Dont make spaces visible
    showtabs=false,                         % Dont make tabls visible
    %columns=flexible,                       % Column format
    %morekeywords={someword, otherword},     % specific keywords
}

\begin{document}
\section*{\large  \"Ubungsblatt 03}
\hrule
\hrule
\vspace{4mm}
%\includegraphics[width=0.8\textwidth]{sampleplot.pdf}
{\bf Aufgabe 1}
Gegeben ist ein Oberfl\"achenpunkt $(1,1,0)^t$ mit Normale $(0,1,0)^t$, eine Lichtquelle im Punkt $(0,4,0)^t$ und der Augenpunkt in $(2,2,0)$.
Berechnen Sie die Helligkeit des  Oberfl\"achenpunktes nach dem Phongschen Beleuchtungsmodell.

\vspace{8mm}
{\bf Aufgabe 2}
Erkl\"aren Sie die Funktionsweise des Shadowmap-Algorithmus zur Darstellung von Schatten.

\vspace{8mm}
{\bf Aufgabe 3}
Erkl\"aren Sie den  Unterschied zwischen Bumpmapping und Displacementmapping.

\vspace{8mm}
{\bf Aufgabe 4}
Erkl\"aren Sie  den Unterschied zwischen Gouraudshading und Phongshading.


\vspace{8mm}
{\bf Aufgabe 5}
Erl\"aultern Sie ein Szenario, bei dem differed Shading schneller sein kann als forward shading und 
 begr\"unden Sie den Geschwindigkeitsgewinn. 

{\bf Aufgabe 6}
Gegeben sind die vier Punkte 
\begin{align*}b_0 := \begin{pmatrix} 0 \\ 0 \\ 0\end{pmatrix}, b_1 :=\begin{pmatrix} 0 \\ 1 \\ 0\end{pmatrix}, b_2 :=\begin{pmatrix} 1 \\ 1 \\ 0\end{pmatrix}, 
b_3 :=\begin{pmatrix} 1 \\ 0 \\ 0\end{pmatrix} 
\end{align*}
und $B(t):=\sum_{i= 0}^{3} H^3_i(t) \cdot b_i $ eine Bezierkurve, die diese als Kontrollpunkte hat. Berechnen Sie mit Hilfe des Algorithmus von De Casteljau
$B(\frac{1}{4})$,$B(\frac{1}{2})$ und $B(\frac{3}{4})$.
 

\vspace{8mm}
{\bf Aufgabe 7}
Erkl\"aren Sie die Rendergleichung und das prinzipielle Vorgehen beim Raytracing.
\end{document}
