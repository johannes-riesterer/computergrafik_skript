\section{Raytracing}

\subsection{Farbwahrnehmung und Farbmodelle}
\subsection{Globale Beleuchtungsmodelle und Rendergleichung}
\subsubsection{BRDF}
Die sogenannte bidirektionale Reflektanzverteilungsfunktion (engl. Bidirectional Reflectance Distribution Function, BRDF)
ist eine Funktion $f_r (x, \omega_i, \omega_j)$, die das Reflexionsverhalten der Oberflächen eines Materials beschreibt.
Sie liefert die Intensität des ausgehenden Lichtes (einer festen Wellenlänge) in Richtung $\omega_j$ am Punkt $x$ bei eingehender Lichtintensität aus Richtung $\omega_i$.
 \begin{figure}[H]
    \centering
    \includegraphics[width=0.6\textwidth]{images/BRDF_Diagram.png}
    \caption{BRDF Funktion}
    \label{fig:raytracin_brdf}
\end{figure}
\subsection{Raycasting}
\subsubsection{"Klassisches" Raytracing}
\subsubsection{Monte Carlo Integration und Pathtracing}
\subsubsection{Datenstrukturen für Bereichsabfragen}

\subsection{Raymarching}
\subsection{Labor}
\subsubsection{Blender}
\subsubsection{Echtzeitfähiges Raymarching in WebGL}
